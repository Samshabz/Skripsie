\section{Engineering Summary}

Global Positioning System (GPS) vulnerabilities, such as jamming and spoofing, pose significant risks to Unmanned Aerial Vehicle (UAV) navigation, potentially leading to loss of control and mission failure. Recognizing the limitations of existing alternatives—which are often prohibitively expensive or unreliable—this project addressed the critical need for a robust, GPS-independent navigation solution for UAVs operating in GPS-denied environments.

The primary objective was to develop an accurate and generalizable image-based GPS localization system. To achieve this, the project optimized the localization pipeline by selecting and integrating effective feature extraction, matching, and planar transformation estimation techniques. The system was designed to estimate GPS locations using only prior telemetry data and image features, ensuring radial errors remained below 10\% of the UAV's displacement from a reference image. Additionally, the system was engineered for real-time operation, achieving response times under two seconds following GPS signal loss, thereby allowing pilots to maintain effective manual control without significant delays. Comprehensive validation across diverse environments confirmed the system's generalizability and robustness, eliminating the need for environment-specific parameter tuning.

The implementation successfully met all outlined objectives. The image-based localization system demonstrated high accuracy and real-time performance across multiple datasets, including challenging sparse-feature environments. It maintained robust accuracy despite practical limitations such as resolution constraints, varying terrain feature densities, and fluctuating environmental conditions. The system's resilience was further evidenced under low-light conditions and scenarios with limited pixel overlap, ensuring reliable navigation data delivery even in adverse situations.

The successful development and validation of this system signify a substantial advancement in UAV navigation technology. By providing a reliable alternative to GPS, the system enhances UAV operational safety and effectiveness in both military and civilian applications. 

However, the project also identified several areas that warrant further research and development. The system was unable to consistently balance the quantity (stability) and quality (outliers) of feature detection across diverse feature landscapes, highlighting the need for more sophisticated algorithms or machine-learning techniques. Additionally, sparse-feature environments in low-light conditions continue to pose significant challenges. Furthermore, this study did not account for distortions resulting from scale and perspective changes, necessitating the application and testing of homography transformations to address these distortions effectively.

In summary, this project successfully developed a versatile and effective image-based GPS localization system for UAVs, addressing the critical need for reliable navigation in GPS-denied environments. The system has proven to be practical for real-world applications, demonstrating robust performance across a wide range of operational conditions. With further enhancements to improve accuracy and overcome existing limitations, this localization system holds substantial promise for enhancing UAV navigation reliability in both military and civilian sectors, supporting a broader scope of missions with increased safety. 