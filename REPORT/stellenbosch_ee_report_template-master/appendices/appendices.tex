\chapter{Project Planning Schedule}
\makeatletter\@mkboth{}{Appendix}\makeatother
\label{appen:derivations_bigramseg}

This is an appendix.

\chapter{Outcomes Compliance}
\makeatletter\@mkboth{}{Appendix}\makeatother
\label{appen:derivations_bigramseg}


\section*{Student’s Graduate Attribute (GA) Achievement Plan}

\subsection*{GA 1: Problem Solving}
The navigation system I developed effectively addresses complex engineering and software problems to achieve a robust and accurate image-based localization system. By leveraging mathematical and statistical methods, I crafted a system that accurately identifies, matches, and estimates the transformations of the UAV even in challenging GPS-denied environments. Throughout the project, I prioritized methods that could reliably extract features and estimate transformations in real time, allowing the UAV to autonomously navigate with precision. I rigorously tested, through over 10,000 lines of code, various algorithms and parameters to ensure high accuracy, outlier rejection, and robustness across different environmental conditions. Furthermore, throughout the project, I faced numerous hurdles regarding issues in the accuracy and runtime of the solution; I had to perform extensive debugging, address prior assumptions, and explore different ways to solve problems. For instance, normalizing images to the global space might seem straightforward, but after noting errors, I realized I had to find their source. Rotational loss was found to be minimized when alignment was applied between images only. Additionally, I used literature to identify innovative approaches that could be adapted to overcome specific challenges related to feature scarcity, low-light conditions, and image warping.

\subsection*{GA 2: Application of Scientific and Engineering Knowledge}
My project extensively utilized computer vision theories, image processing techniques, and pre-trained machine learning models to develop an effective navigation system. I applied various algorithms grounded in computer vision to accurately detect and analyze landscape changes. Leveraging principles of homographic transformations, I was able to account for the UAV’s motion and orientation without relying on external GPS signals. Furthermore, I incorporated statistical methods to quantify model output and to improve accuracy by systematically testing different outlier rejection methods. This application of scientific and engineering knowledge was instrumental in developing a system that could operate reliably across multiple terrains, validating the project’s approach to image-based navigation.

\subsection*{GA 3: Engineering Design}
The engineering design process began with creating a high-level flow diagram that outlined each stage in the image processing pipeline. This design incorporated multiple resources and techniques and integrated them effectively with robust design measures to handle inaccuracies. I utilized my knowledge of engineering principles gained during the course of my studies in design skills and software development to design a modular, robust system capable of adapting to different environments. This structured approach allowed the system to maintain stability and precision, even in scenarios with sparse image features or environmental variations, ultimately fulfilling the project's goals of reliable UAV navigation.

\subsection*{GA 4: Investigations, Experiments, and Data Analysis}
To support the development and validation of the system, I collected extensive data across various environments and under different conditions. I systematically tested the navigation system using diverse datasets and conditions, such as sparse desert landscapes and low-light repetitive terrain, to identify the suitability, generalizability, accuracy, and robustness of different techniques. Using statistical analysis, I was able to quantify the variance in results, leading to the subsequent discovery of faults in logic and techniques, as well as understanding the true performance of the system. The ability to leverage systematic investigation and experimentation, as well as my knowledge of data analytics, allowed me to implement, test, and optimize various methods with limited guidance. The project required ongoing experimentation with different feature detection and matching algorithms, including neural network-based methods, to fine-tune the system's performance. The large array of parameters needed to be tuned meant I could not simply test every possible solution; I needed to systematically investigate from principles and then experiment based on those to ensure efficiency and not forgo a better alternative.

\subsection*{GA 5: Engineering Methods, Skills, and Tools (including IT)}
I employed a wide range of tools and engineering methods, including Python libraries such as OpenCV and advanced frameworks available on GitHub for image processing. Furthermore, this project involved rigorous research into the best and most current solutions for all stages of the pipeline. By using the latest methods, I was able to demonstrate the applicability of this system in difficult conditions, something made possible by recent advancements in computer vision techniques. This project also required rigorous adherence to software engineering practices, ensuring modularity, code optimization, and thorough error handling. I adopted computational techniques to evaluate the system's accuracy and performance, allowing for the assessment and validation of methods in a controlled environment. By following best practices and leveraging tools to address issues in real time, I was able to identify areas for improvement and continuously optimize the system’s accuracy and response.

\subsection*{GA 6: Professional and Technical Communication}
Throughout the project, I demonstrated strong communication skills by preparing both written reports and oral presentations. These presentations communicated complex technical concepts in a clear and concise manner to stakeholders and supervisors. Additionally, I documented my methods and findings in detail, ensuring reproducibility and clarity for future work. The project’s outcomes were communicated effectively in both technical and non-technical formats, highlighting my ability to translate technical progress and results into comprehensible information.

\subsection*{GA 8: Individual Work}
As the lead on this project, I took primary responsibility for all aspects of the development and implementation process. This included problem-solving, design, and optimization, as well as the testing and analysis of results. My role required me to work independently to research, develop, and refine solutions, utilizing various resources and overcoming challenges without external assistance. This project reflects my commitment to delivering high-quality work and achieving the outlined objectives independently.

\subsection*{GA 9: Independent Learning Ability}
This project demanded a high level of independent learning to solve complex navigation challenges in GPS-denied environments. I actively sought out research materials, consulted technical documentation, and applied knowledge from relevant scientific literature. I expanded my understanding of image processing techniques, statistical analysis, and feature extraction methods independently, equipping myself with the skills necessary to achieve project goals without requiring extensive guidance. This process underscored my ability to learn independently and apply new knowledge effectively to overcome project-specific challenges.
