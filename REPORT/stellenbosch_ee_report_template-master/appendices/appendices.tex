% \chapter{Project Planning Schedule}
% \makeatletter\@mkboth{}{Appendix}\makeatother
% \label{appen:Images}

% This is an appendix.
% add pics. 

\chapter{Outcomes Compliance}
\makeatletter\@mkboth{}{Appendix}\makeatother
\label{appen:GradAttribute}


\section*{Student’s Graduate Attribute (GA) Achievement Plan}

\subsection*{GA 1: Problem Solving}
I have met this objective by taking a broadly defined problem: developing an image-based GNSS-independent navigation system for UAVs, and systematically solving this through research, synthesis of an optimal pipeline, experimentation and analysis. Using mathematical and statistical methods, I created a system that identifies, matches, and estimates UAV transformations in real time. Further, this involved investigating the sources of pipeline errors and solving them via improving understanding of fundamental global positioning concepts. Finally, this involved the numerous software engineering challenges that required constant problem-solving to ensure logical soundness, robustness and accuracy.

\subsection*{GA 2: Application of Scientific and Engineering Knowledge}
My project utilized computer vision theories, image processing techniques, and pre-trained machine learning models. I applied algorithms to detect and analyze landscape changes, leveraging homographic transformations to account for relative UAV motion without GNSS signal. Statistical methods improved accuracy by testing different outlier rejection methods, ensuring reliable operation across multiple terrains.

\subsection*{GA 3: Engineering Design}
I developed a high-level flow diagram for the image processing pipeline, integrating multiple resources and techniques with robust measures to handle inaccuracies without detailed pipelines available in literature. My engineering principles and design skills ensured development of a system that was built from the ground up to be adapt to various, dense and sparse, environments without manual intervention, maintaining stability and precision. This design also ensures modular design to facilitate scalability and future enhancements.

\subsection*{GA 4: Investigations, Experiments, and Data Analysis}
I collected extensive data across various environments and conditions, systematically testing the navigation system to identify accuracy and robustness. Statistical analysis ensured accurate quantification of errors, such as knowing how to best quantify system performance. For instance, this included looking at maxima of error, quantifying them in different metrics for different perspectives, and analyzing potential sources of errors based on nuanced differences such as variance in ground image overlap. Ongoing experimentation with feature detection and matching algorithms, including neural network-based methods, fine-tuned the system's performance.

\subsection*{GA 5: Engineering Methods, Skills, and Tools (including IT)}
I employed various Tools, including Python libraries like OpenCV, Matplotlib, Sklearn for data analysis and advanced frameworks for image processing. Rigorous research into current solutions and adherence to software engineering practices and engineering methods ensured systematic solving of parts of the problem, modularity, code optimization, and error handling. Different computational techniques allowed for precise text and visual feedback for evaluation and debugging, allowing continuous optimization.

\subsection*{GA 6: Professional and Technical Communication}
This project involves both an oral presentation and written report; thereby demonstrating my ability to communicate effectively, both orally and in writing. Further, this project involved using diagrams and tables to communicate complex concepts concisely, facilitating understanding of the project's technical aspects.

\subsection*{GA 8: Individual Work}
I took responsibility for all aspects of the development and implementation process. This included problem-solving, design, and optimization, as well as the testing and analysis of results. My role required me to work independently to research, develop, and refine solutions, utilizing various resources and overcoming challenges with weekly feedback from my supervisor.

\subsection*{GA 9: Independent Learning Ability}
This project demanded independent learning to solve complex navigation challenges in GNSS-denied environments. This was done through researching different research materials to gain an understanding and relevant technical studies. I expanded my understanding of Computer Vision independently, equipping myself with the skills necessary to achieve project goals without requiring extensive guidance.
