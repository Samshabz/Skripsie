
\chapter{Literature Review}


This chapter provides a comprehensive review of the existing literature on UAV navigation systems, focusing on image-based approaches and alternative navigation methodologies. The discussion encompasses feature extraction, matching, and planar transformations, elucidating the theoretical underpinnings essential for the proposed image-based navigation system. By exploring the strengths and limitations of various navigation techniques, this review sets the stage for the subsequent system design and implementation, highlighting the critical role of feature-based methods in UAV navigation.

% -----------------------------------------------------------------------------------------------------------------------------------


\section{Global Navigation Satellite Systems (GNSS) Vulnerabilities}


Global Navigation Satellite Systems (GNSS), including the U.S. Global Positioning System (GPS), Russia's GLONASS, the European Union's Galileo, and China’s BeiDou, provide essential positioning, navigation, and timing (PNT) information globally. GNSS operates by deploying a constellation of satellites that transmit precise time-stamped signals to Earth-based receivers. Each receiver calculates its position by measuring the time delay of signals from multiple satellites, triangulating data from at least four satellites to determine its three-dimensional location. While GNSS is critical across sectors such as aviation, maritime, and land navigation, its reliance on weak, line-of-sight satellite signals introduces significant vulnerabilities \cite{geotab2024gps}.

The reliance on GPS for UAV navigation has grown extensively, yet vulnerabilities to jamming and spoofing have become increasingly pronounced. With the advent of more accessible jamming and spoofing technology, intentional GPS interference is no longer a complex undertaking, raising serious concerns for security-sensitive applications. Instances of GPS jamming and spoofing are particularly prevalent in conflict zones, where adversaries exploit GPS weaknesses to disrupt or mislead UAV operations. The issue is far from hypothetical; In fact, more than 1,100 incidents of GPS spoofing alone were reported worldwide in August 2024, underscoring the urgency of addressing these vulnerabilities \cite{khalil2024gnss}.

Such interference has direct implications for national security, as border surveillance and reconnaissance missions are heavily dependent on accurate GPS data \cite{Weiss2024}. Consequently, alternative redundancy navigation methods that are not susceptible to such interference are critical in ensuring operational continuity and resilience in high-stakes scenarios.


\section{Alternative Navigation Approaches}

In the realm of UAV navigation, various redundancy measures have been considered and implemented to reduce reliance on GNSS. This section reviews these techniques, highlighting their strengths and limitations in the context of UAV applications.

\subsection{Quantum Sensing Navigation}

Quantum sensing navigation employs cold atom inertial sensors to achieve superior precision compared to traditional inertial measurement units (IMUs). In this approach, cold atoms are trapped in a vacuum and manipulated using lasers to measure motion through interference patterns. This method offers exceptionally accurate acceleration and rotation data, theoretically minimizing navigation drift \cite{wright2022cold}.

Despite its high precision, quantum sensing navigation presents significant challenges. The primary drawbacks include:

\begin{itemize}
    \item \textbf{Cost and Size}: Cold atom sensors are expensive and bulky, making them impractical for most UAV applications where cost-efficiency and compactness are crucial.
    \item \textbf{Power Consumption}: These sensors require substantial power, limiting their feasibility for battery-operated UAVs.
    \item \textbf{Operational Limitations}: Although aviation-grade systems incorporating cold atom sensors can reduce drift by approximately half, they still suffer from significant drift over extended periods. Additionally, sensor misalignment and noise, especially when gyroscope noise surpasses thresholds like 0.05, 0.10, or 0.20 rad/s at higher frequencies, can exacerbate errors \cite{wright2022cold}.
\end{itemize}

These factors collectively restrict the adoption of quantum sensors in UAV navigation systems, particularly those that are cost-sensitive and size-constrained.

\subsection{Odometry-Based Solutions}

Odometry-based navigation estimates UAV displacement by tracking wheel rotation or utilizing accelerometer readings. This method provides a direct measurement of movement, allowing for real-time tracking of the UAV's position.

However, odometry systems are inherently susceptible to cumulative drift, where minor errors accumulate over time and distance, leading to significant inaccuracies in long-range navigation \cite{Zhuang2023}. Specifically:

\begin{itemize}
    \item \textbf{Error Accumulation}: Small discrepancies in sensor measurements can accumulate, resulting in substantial positional errors over extended flights.
    \item \textbf{Environmental Factors}: Variations in terrain, wheel slippage, or unexpected accelerations can further degrade odometry accuracy.
\end{itemize}

For UAVs, these limitations render odometry-based methods unsuitable as a primary navigation solution for long-distance missions. Nevertheless, odometry can be effectively utilized for short-range navigation tasks, such as pathfinding and supplementary guidance in conjunction with more reliable systems like image-based navigation.


\subsection{RF Communication and Signals-Based Navigation}

RF communication systems enable navigation by triangulating the UAV’s position using signal sources like ground beacons, cellular towers, or Wi-Fi networks. By analyzing signal strength, frequency, and timing, RF-based navigation offers an independent alternative to GPS, particularly viable in urban areas with abundant signals and enhanced security through encryption. However, reliance on infrastructure limits its use in remote areas, and the Earth's curvature further restricts line-of-sight over long distances or at lower altitudes, reducing signal strength and accuracy \cite{brewer_line_2024}.


\subsection{LIDAR and Radar for Ground Mapping}

LIDAR and radar are effective for UAV ground mapping through active wave emission, measuring distances to surfaces and generating detailed 3D environmental maps. These sensors are advantageous for precise altitude measurements and mapping, allowing for accurate location inference. However, LIDAR and radar systems emit detectable signals, which may be undesirable for applications requiring minimal detectability. Additionally, these systems can be resource-intensive and exhibit limitations in dense environments, where interference and signal scattering may impair accuracy and operational range \cite{scoutaerial2024lidar}.

\subsection{Image-Based Homography Navigation}

Image-based homography navigation leverages homography estimation techniques to enhance UAV navigation by aligning images and inferring planar transformations. This approach uses feature extraction and matching from reference images to determine the UAV’s orientation and position relative to its environment. By creating a reliable alignment between the UAV’s current view and previously captured data, the system can accurately infer location and heading, enabling effective navigation back to base, particularly in GPS-denied or GPS-compromised scenarios. In this application, reference images are captured during flight before any GPS loss and are later used to guide the UAV’s return to base. Future applications could extend this method by utilizing a database of recent reference images, allowing broader deployment across various terrains and operational contexts.


This method addresses several limitations associated with alternative navigation solutions:

\begin{itemize}
    \item \textbf{Cost and Size}: Unlike quantum sensing systems, image-based navigation relies on standard imaging hardware, which is typically more cost-effective and compact, making it suitable for UAV applications where size and budget constraints are critical.
    \item \textbf{Stealth}: Unlike LIDAR and radar, which emit detectable signals, image-based navigation does not "announce" the UAV's presence, allowing for lower detectability in sensitive or security-focused applications.
    \item \textbf{Power Consumption}: Image processing can be optimized for low power consumption compared to the high energy demands of cold atom sensors, thereby enhancing the feasibility for battery-operated UAVs.
    \item \textbf{Cumulative Drift and Error Accumulation}: While odometry-based systems suffer from cumulative drift over time, the usage of this device specifically to navigate back to base ensures constant correction and is thereby not affected by drift. 
    \item \textbf{Environmental Adaptability}: The proposed system is designed to handle environmental variability by incorporating robust feature detection and noise filtering, ensuring reliable performance across diverse operational contexts.
    \item \textbf{Real-Time Processing}: The implementation focuses on real-time image processing and transformation estimation, enabling immediate adjustments to the UAV’s navigation path and enhancing overall responsiveness.
\end{itemize}


Much research exists in the field of homography, including its application to aerial imagery, as demonstrated in \cite{Zhang2024}, where its effectiveness across various applications is well established. However, existing studies primarily focus on high-level overviews or isolated improvements within the homography algorithm. This study distinguishes itself by not only evaluating the effectiveness of homography estimation for UAV navigation but also by providing a comprehensive implementation pipeline, encompassing feature detection through to transformation estimation, offering a practical framework that can be readily adapted for diverse applications.

Image-based navigation systems are often overlooked due to perceived limitations in achieving real-time performance, being robust to distortions, generalizing across various environments, and attaining high accuracy. However, as computer vision, machine learning techniques, and camera technology continue to advance, these challenges are becoming increasingly surmountable. Their main practical limitation is related to extreme visibility issues, which are relatively infrequent. This study aims to develop an optimized model pipeline and evaluate the viability of the image-based solution in real-world scenarios, determining its potential to be accurate, efficient, and adaptable across diverse environments.




% -----------------------------------------------------------------------------------------------------------------------------------

\section{Background}

This section provides a comprehensive overview of the fundamental concepts and techniques that underpin the proposed image-based UAV navigation system. By delving into feature extraction, matching, and planar transformations, it establishes the theoretical foundation essential for the subsequent system design and implementation. The discussion emphasizes the critical role of feature-based methods over direct approaches, setting the stage for understanding the chosen methodologies.


% -----------------------------------------------------------------------------------------------------------------------------------


\subsection{Planar Transforms}

Planar transformations are mathematical operations that map points from one plane to another, accounting for changes in position, orientation, and perspective. In the context of UAV navigation, the output of these transformations are directly used to infer the UAV's location and heading based on prior reference images and telemetry data.

Two primary approaches exist for estimating planar transforms: direct methods and feature-based methods, each with distinct strengths and limitations \cite{nguyen2018}. 
\subsubsection{Direct Methods}

Direct, or global, methods utilize the entire image to estimate planar transforms by comparing pixel intensities and minimizing the differences through optimization techniques such as gradient descent \cite{nguyen2018}. Key characteristics include:

\subsubsection{Direct Methods}

Direct methods, also known as global methods, utilize the entire image to estimate planar transformations by comparing pixel intensities and minimizing differences through optimization techniques such as gradient descent \cite{nguyen2018}. Key characteristics of direct methods include:

\begin{itemize}
    \item \textbf{Robustness for Small Viewpoint Changes}: These methods perform well when dealing with minor adjustments in viewpoint, maintaining accuracy during incremental movements.
    \item \textbf{Sensitivity to Noise}: Direct methods are highly susceptible to noise from defective pixels or environmental factors, which can degrade transformation accuracy.
    \item \textbf{Assumption of Smooth Movements}: They rely on the premise of smooth, incremental UAV movements. Large changes in rotation, perspective, or translation, such as those encountered when a UAV turns back to base, render direct methods unreliable \cite{nguyen2018}.
    \item \textbf{Similarity Comparison}: Direct methods evaluate image similarity based on the global context, which is beneficial for understanding overall similarity between images.
    \item \textbf{Computational Stability}: Although direct methods involve pixel-wise comparisons that may be seem computationally expensive, the relatively simple and single layer computations allow for efficient and stable computations across different operational conditions \cite{nguyen2018}. In contrast, local methods require multiple stages of processing and become significantly slower if the number of features becomes too large.
\end{itemize}

\subsubsection{Feature-Based Methods}

In contrast, feature-based methods extract and match keypoints from both reference and real-time images, often employing multiple stages of outlier removal \cite{GlobalLocal2023}. Key advantages of feature-based methods include:

\begin{itemize}
    \item \textbf{Suitability for Large Transformations}: Feature-based methods excel in scenarios involving significant viewpoint changes and rotations, as they focus on distinctive features rather than pixel-wise comparisons.
    \item \textbf{Computational Efficiency}: By targeting salient features, these methods can perform multiple stages of processing without having to analyze every pixel, thereby enhancing computational efficiency. However, care must be taken to ensure that the number of features does not become excessive, as this can lead to performance degradation.
    \item \textbf{Robustness to Distortions}: Feature-based methods are resilient to various forms of distortion and environmental variability, ensuring reliable performance across different operational conditions.
\end{itemize}


% -----------------------------------------------------------------------------------------------------------------------------------


\subsection{Feature Detectors}

Feature extraction is a cornerstone of image-based UAV navigation, enabling the estimation of transformations such as rotation and translation between consecutive images. Feature detectors identify \textbf{keypoints}—distinct, repeatable points within an image—and generate \textbf{descriptors} that encapsulate information about the local image region surrounding each keypoint. These keypoints and descriptors are essential for accurate matching across multiple frames, allowing the system to track movement while maintaining invariance to changes in scale, rotation, and illumination. This precision is critical for accurately inferring both rotational and translational shifts between images.

In this context, \textit{features} refer to both keypoints and descriptors, as each feature extraction method provides both components to facilitate effective image matching. The selection of feature extractors is guided by criteria such as accuracy, computational efficiency, robustness to diverse datasets, and the necessity for freely available tools, thereby excluding proprietary methods like SURF due to licensing fees. Additionally, chosen methods must hold credibility within the field, given their application in safety-critical UAV navigation systems.

The feature detectors selected for this system—\textbf{ORB}, \textbf{AKAZE}, and \textbf{SuperPoint with LightGlue}—offer a range of performance characteristics, balancing accuracy, computational efficiency, and machine learning integration. These extractors are further elaborated in the subsequent sections.

\subsubsection{ORB (Oriented FAST and Rotated BRIEF)}

\textbf{ORB} combines the \textbf{FAST} keypoint detector with the \textbf{BRIEF} descriptor, enhanced for rotation invariance. \textbf{FAST} rapidly identifies keypoints by analyzing pixel intensity differences in a circular region around each candidate point. Once detected, \textbf{BRIEF} encodes the local image patch into a binary string through intensity comparisons. ORB introduces rotational invariance by aligning keypoints based on their dominant orientation before descriptor computation. This enhancement makes ORB both fast and robust to scale and in-plane rotation, although it may struggle with repetitive textures or complex lighting variations \cite{opencv_orb_tutorial}.

\subsubsection{AKAZE (Accelerated-KAZE)}

\textbf{AKAZE} constructs a nonlinear scale space using diffusion-based filtering, capturing finer image details more effectively than linear methods. It detects keypoints by assessing local contrast with a specialized adaptive filter, enabling the identification of subtle features that simpler detectors might miss. The \textbf{Modified Local Difference Binary (MLDB)} descriptor encodes the neighborhood of each keypoint into a binary vector based on pixel intensity differences. While AKAZE is both fast and compact, its performance can be sensitive to detection thresholds across different environments, potentially affecting its robustness in varied operational contexts \cite{opencv_akaze}. 

\subsubsection{SuperPoint with LightGlue}

\textbf{SuperPoint} is a deep learning-based keypoint detector and descriptor that leverages convolutional neural networks (CNNs) to identify and describe keypoints in a single forward pass. Pre-trained on extensive image datasets, SuperPoint excels at recognizing stable and distinctive keypoints under varied conditions \cite{rpaultrat2023superpoint}. However, its performance may degrade on datasets significantly different from its training data. Pairing SuperPoint with \textbf{LightGlue}, a machine-learning-based matcher, enhances matching accuracy through advanced graph-based techniques \cite{cvg2023lightglue}. Despite their high accuracy, SuperPoint and LightGlue are computationally intensive, necessitating GPU acceleration for real-time applications.



% -----------------------------------------------------------------------------------------------------------------------------------


\subsection{Feature Matching}

Feature matching establishes correspondences between keypoints in different images based on descriptor similarity. After identifying these correspondences, ambiguities and low-quality matches are removed to retain only the most reliable matches, which are then used to estimate transformations such as translation and rotation. This filtering ensures that transformation estimations are based solely on mutual information between images, enhancing the accuracy and reliability of the navigation system.

Each matcher generates a list of potential matches along with their similarity scores, quantified using a descriptor-space distance metric. These scores are instrumental in determining the quality of the matches and play a crucial role in the subsequent filtering and transformation estimation processes.

\subsubsection{Types of Feature Matchers}

Two primary feature matchers are employed in this system: the \textbf{Brute-Force Matcher (BFMatcher)} and the \textbf{Fast Library for Approximate Nearest Neighbours (FLANN)}. Additionally, \textbf{LightGlue}, a machine-learning-based matcher, is utilized for its enhanced accuracy despite its higher computational demands.

\begin{itemize}
    \item \textbf{Brute-Force Matcher (BFMatcher)}: The Brute-Force Matcher compares each feature in one image with every feature in the second image, ensuring the best possible match based on descriptor similarity. While this guarantees high accuracy, it is computationally expensive, especially with large numbers of keypoints, making it less suitable for real-time applications without optimization \cite{opencv_bfmatcher}.
    
    \item \textbf{Fast Library for Approximate Nearest Neighbours (FLANN)}: FLANN accelerates the nearest neighbour search in high-dimensional descriptor spaces using algorithms such as KD-trees or hierarchical clustering. This approximate matching approach offers significant speed improvements with minimal loss in accuracy, making it ideal for real-time applications with extensive datasets \cite{opencv_flann_tutorial}
    
    \item \textbf{LightGlue}: Leveraging deep learning, LightGlue improves matching accuracy by employing advanced graph-based techniques to establish more reliable correspondences. Although highly effective, its structure necessitates the use of other neural network-based feature extractors like SuperPoint to realize its full potential \cite{cvg2023lightglue}. The enhanced accuracy comes at the cost of increased computational demands, requiring GPU acceleration for optimal performance.
\end{itemize}

\subsubsection{Search Techniques}

The search technique determines which potential matches are retained for further processing. Various methods, including \textbf{Radius Search}, \textbf{Vanilla Matching}, and \textbf{K-Nearest Neighbours (KNN) Matching}, are explored to control the number and quality of matches.

\begin{itemize}
    \item \textbf{Radius Search}: This method retains matches within a specified distance in descriptor space, effectively filtering out weaker matches. However, it does not guarantee a fixed number of matches per keypoint, leading to inconsistent results \cite{opencv_matcher_tutorial}. 
    
    \item \textbf{K-Nearest Neighbours (KNN) Matching}: KNN matching retains the top \( K \) matches for each keypoint, allowing the application of post-filtering techniques such as Lowe’s ratio test to eliminate ambiguous matches \cite{opencv_matcher_tutorial}.
    
    \item \textbf{Vanilla Matching}: Vanilla matching returns the single best match for each keypoint based on the closest descriptor distance. It is a subset of KNN matching with \( K=1 \), offering simplicity and ease of implementation \cite{opencv_matcher_tutorial}. 
    

\end{itemize}



% -----------------------------------------------------------------------------------------------------------------------------------




\subsection{Image Similarity Computation}

Image similarity computation is a pivotal component of UAV navigation systems that rely on reference images for accurate localization and pose estimation. Effective similarity measures ensure efficient processing of extensive image datasets and facilitate precise transformation estimations, which are essential for reliable navigation.

\subsubsection{Proximity-Based Techniques}

To achieve real-time performance, the search space is reduced to images within the proximity of UAV's last known location, filtering images within a static or dynamic radius. While this method is highly efficient, it does not account for potential deviations from the expected flight path or the presence of poor-quality reference images. This limitation implicates that this measure cannot be used as the sole basis for image similarity computation, necessitating the integration of additional techniques for comprehensive assessment.

\subsubsection{Global Matching Techniques}

Global matching techniques evaluate image similarity based on the overall context of the images. Although these methods are less precise for pose inference compared to local feature matchers, their ability to capture the holistic visual information makes them effective for similarity comparisons. Additionally, their computational efficiency is advantageous for processing large image datasets. These methods necessitate rotational alignment of images to eliminate bias introduced by orientation discrepancies \cite{GlobalLocal2023}.

Several global matching techniques are employed to score image similarity:

\textbf{Cross-Correlation}

Cross-correlation measures similarity by sliding one image over another and computing the sum of pixel-wise multiplications at each position. The peak value signifies the best alignment, and its magnitude indicates the confidence level of the similarity. Higher confidence values reflect greater similarity between the images. While straightforward to implement, cross-correlation is sensitive to noise and illumination changes, which can compromise the reliability of the similarity measure \cite{sharma2022crosscorrelation}.

\textbf{Histograms}

Histogram comparison assesses similarity by analyzing the distribution of pixel intensities within each image. Typically, each image's histogram is divided into 256 intensity bins, and similarity is quantified using metrics such as Chi-Square or Bhattacharyya distance. This method emphasizes global color and brightness distributions but neglects spatial information, making it less effective for nuanced structural differences \cite{rosebrock2014comparehistograms}.

\textbf{Structural Similarity Index (SSIM)}

SSIM evaluates similarity by decomposing images into luminance, contrast, and structure components. It computes local statistics within small windows and integrates them into a single similarity score that mirrors perceived image quality. SSIM effectively captures structural information like edges and textures, aligning closely with human visual perception. Although slightly more computationally expensive than the former methods, it is robust to varied conditions \cite{rosebrock2017imagedifference}.

\textbf{Local Detectors Conversion}

Although not inherently a global matching technique, local feature matching can be adapted to achieve a global understanding of image similarity. This involves identifying and matching keypoints in both images and assessing the overall number of good matches. However, this approach alone does not ensure an even distribution of matches across the entire image, potentially leading to biased, localized similarity assessments. To mitigate this, a grid matching technique is employed, dividing the image into grids and limiting the number of matches per grid. Although the most computationally intensive, this method enhances robustness against distortions and rotations by ensuring a uniform distribution of matches across the image.



% -----------------------------------------------------------------------------------------------------------------------------------


\subsection{Planar Transformation Estimators}

Planar transformation estimators are pivotal for calculating the UAV's changes in position and orientation between consecutive images. These feature-based transformations enable precise updates to the UAV's pose and location, facilitating accurate navigation and control. The following subsections outline the primary planar transformations employed in this system.

\subsubsection{Affine Transformation}

Affine transformation captures translation, rotation, scaling, and shear, providing six degrees of freedom. It is represented by a  \(2 \times 3\) matrix that maps points from one plane to another while preserving lines and parallelism. Affine transformations are computed by estimating the affine transformation matrix between two sets of corresponding points using OpenCV's \texttt{estimateAffine2D} function \cite{opencv_warp_affine}. While versatile, the inclusion of scaling and shear can introduce unnecessary complexity for scenarios where only rotation and translation are relevant, potentially impacting the accuracy of transformation estimations.

\subsubsection{Rigid Transformation Estimation (SVD)}

Rigid transformation preserves the shape and size of objects by estimating only rotation and translation, excluding scaling and shear. Represented by a  \(2 \times 3\) matrix, rigid transformation ensures orthogonality in the rotation component. Utilizing Singular Value Decomposition (SVD), this method minimizes the least-squares error between two point sets. The process involves:

\begin{enumerate} 
    \item Computing the weighted centroids of both point sets. 
    \item Centering the points by subtracting their respective centroids. 
    \item Calculating the covariance matrix of the centered points. 
    \item Performing SVD on the covariance matrix to derive the rotation matrix. 
    \item Determining the translation vector based on the centroids. 
\end{enumerate}

The resulting \(2 \times 2\) rotation matrix and \(2 \times 1\) translation vector are combined to form the rigid transformation matrix. Rigid transformation is computationally efficient and well-suited for applications requiring only rotation and translation \cite{sorkine2017least_squares}.

\subsubsection{Partial Affine Transformation}

Partial affine transformation simplifies the full affine model by focusing solely on translation, rotation, and uniform scaling, offering four degrees of freedom. This transformation is also represented by a \(2 \times 3\) matrix, similar to the affine transformation but without shearing and with reduced, uniform scaling. By limiting scaling to be uniform, partial affine transformation reduces potential distortions and maintains simplicity, making it ideal for scenarios where only rotation and translation are significant \cite{opencv_warp_affine}.

\subsubsection{Homography Transformation}

Homography transformation accounts for translation, rotation, scaling, shear, and perspective distortion, providing eight degrees of freedom. It is represented by a \(2 \times 3\) matrix and is estimated using OpenCV's \texttt{findHomography} function, typically with RANSAC for outlier rejection \cite{opencv_homography}. While homography offers greater flexibility in modeling complex transformations, its additional degrees of freedom can introduce unnecessary errors and computational overhead for simpler applications that only require rotation and translation. Consequently, homography is reserved for scenarios necessitating comprehensive transformation modeling beyond rotation and translation.


% -----------------------------------------------------------------------------------------------------------------------------------

\subsection{Optimization Techniques}

Optimizing parameter sets is crucial for enhancing the performance and robustness of the UAV navigation system. These optimization techniques aim to refine the accuracy and reliability of the point cloud used for transformation estimation by effectively filtering out erroneous matches and improving transformation accuracy. The following subsections detail the primary optimization methods employed in this system.

\subsubsection{Random Sample Consensus (RANSAC) for Planar Transformation}

RANSAC is a robust estimation technique used to estimate planar transformations by iteratively selecting random subsets of point correspondences to fit a model and identify inliers \cite{ransac1981random}. The process involves:

\begin{enumerate} 
    \item Randomly selecting a minimal subset of point pairs. 
    \item Estimating the transformation model (e.g., affine or homography) based on the selected subset. 
    \item Determining the number of inliers that fit the estimated model within a predefined threshold. 
    \item Repeating the process for a set number of iterations or until a sufficient inlier ratio is achieved. 
\end{enumerate}

This approach is highly effective in datasets with significant outliers, as it focuses on finding a model that best fits the largest subset of inliers. However, due to its iterative nature and the need to sample repeatedly, RANSAC can result in increased runtime, particularly in larger datasets or when dealing with numerous outliers \cite{ransac1981random}.

\subsubsection{Local Maxima Extrema Density Selection (LMEDS) for Planar Transformation}

LMEDS is a keypoint selection method designed to prioritize areas of high feature density by identifying local maxima as keypoints for matching \cite{lmeds2002local}. This technique involves:

\begin{enumerate} 
    \item Analyzing the image to identify regions with high feature density. 
    \item Selecting keypoints located at local maxima within these dense regions. 
    \item Filtering out less significant keypoints to reduce redundancy and improve match quality.
\end{enumerate}

By concentrating on areas with high feature concentration, LMEDS ensures that keypoints represent the most distinctive and informative regions of the image, enhancing both accuracy and performance by reducing redundant keypoints and improving match quality, particularly in areas with high feature variability. This method is effective in outlier rejection, leading to more reliable transformation estimations.

\subsubsection{Lowe's Ratio Test}

Lowe's ratio test is a filtering technique used to eliminate ambiguous or false keypoint matches by comparing the distance of the best match to the second-best match \cite{lowe2004}. The procedure involves:

\begin{enumerate} 
    \item For each keypoint match, calculating the ratio of the distance of the best match to that of the second-best match. 
    \item Retaining the match if this ratio is below a predefined threshold (commonly 0.75). 
\end{enumerate}

A lower ratio indicates that the best match is significantly better than the alternatives, thereby increasing the likelihood of the match being correct. This test is essential for ensuring that only the most reliable correspondences are used in subsequent processing stages, thereby enhancing the overall accuracy of the transformation estimation.

\subsubsection{Parallel Flow Filtering}

Parallel Flow Filtering ensures consistency in feature matches by aligning them with the average motion flow resulting from the UAV's rotation and translation. In typical UAV movements, feature flow is linear and parallel to the direction of motion, assuming no scaling as per the system's scope. This technique involves:

\begin{enumerate}
    \item Calculating an average motion vector by combining mean and median motion vectors derived from the keypoint matches. 
    \item Filtering out matches that deviate beyond a specified angular threshold from the average motion vector.
\end{enumerate}

By removing matches that do not conform to the expected motion pattern, Parallel Flow Filtering ensures that only consistent and reliable matches contribute to the transformation estimation. This significantly improves the overall accuracy and reliability of the navigation system by maintaining alignment with the UAV's actual motion dynamics.

\subsubsection{n-Match Thresholding}
This method involves setting a threshold that allows only a specific number of matches with the smallest descriptor distances to be retained. This approach ensures that only the most confident matches are considered, reducing the impact of weaker or less reliable matches.

\subsubsection{Absolute Thresholding of Matches}

Absolute thresholding filters matches based on a fixed distance in descriptor space. Only matches that meet or fall below this predefined distance threshold are retained, ensuring that only sufficiently similar matches are used in the transformation estimation process.

