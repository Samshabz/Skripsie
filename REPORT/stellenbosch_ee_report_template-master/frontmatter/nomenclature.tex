\chapter*{Nomenclature\markboth{}{Nomenclature}}
\addcontentsline{toc}{chapter}{Nomenclature}

% \vspace*{-3mm}
\subsubsection*{Variables and Functions}

\begingroup
\renewcommand{\arraystretch}{1.2}
\renewcommand{\tabularxcolumn}[1]{p{#1}}
\begin{tabularx}{\textwidth}{@{}p{2.5cm}L}
    $RMSE$ & Root Mean Square Error, representing the average magnitude of errors. In this study, the usage of this metric specifically, and frequently, uses this term to refer to the radial error in GPS estimate, in metres, averaged across the dataset. \\
    $d_{\text{radial}}$ & Radial error distance, measuring the distance between estimated and true positions in meters.\\
    $x$, $y$ & Coordinates of a point in an image or on the ground plane.\\
    $\theta$ & Rotation angle between consecutive images in degrees or radians, used for orientation alignment.\\
    $T$ & Transformation matrix representing the planar transformation between images.\\
    $H$ & Homography matrix, specifically mapping points from one image plane to another.\\
    $GPS$ & Global Positioning System, providing geolocation and time information for navigation.\\
    $GNSS$ & Global Navigation Satellite System, a generic term for satellite-based navigation systems.\\
    $MAE$ & Mean Absolute Error, measuring the average localization error without directionality.\\
    $MI$ & Mutual Information, measuring the similarity or overlap between images, often used in image matching.\\
    Lat-Lon Estimation & Estimation of Latitude and Longitude; If Referencing an error, it is the radial one.\\
\end{tabularx}
\endgroup

\newpage
\subsubsection*{Key Terms and Concepts}

\begingroup
\renewcommand{\arraystretch}{1.2}
\begin{tabularx}{\textwidth}{@{}p{2.5cm}L}

    \textbf{Features} & 
    Unique keypoints in an image along with their descriptors, used for identifying and uniquely matching points across different images for localization. \\
    
    \textbf{Affine Transformation} & 
    A planar transformation that preserves points, straight lines, and planes, including translation, scaling, rotation, and shearing. \\

    \textbf{Descriptors} & 
    Numerical values describing the characteristics of keypoints in an image, allowing for effective matching between different images. \\

    \textbf{Feature Detectors (or Extractors)} & 
    Processes for identifying distinct features in an image, which are invariant to changes in scale, rotation, and illumination. \\

    \textbf{Global Matching} & 
    Process of finding similarities and determining pose between entire images, considering full-image context rather than isolated keypoints. \\

    \textbf{GPS (Global Positioning System)} & 
    A satellite navigation system providing geolocation and time information. Vulnerable to jamming and spoofing, posing reliability issues for UAV navigation. \\

    \textbf{Homography} & 
    A planar transformation mapping points from one image plane to another in 2D space. Used in image processing to describe transformations like rotation, translation, scaling, and shearing. \\

    \textbf{Homography Estimation} & 
    The process of calculating the homography matrix that defines the transformation between two images for alignment and reliable localization. \\

    \textbf{Keypoints} & 
    Specific points in an image used to identify features. Typically areas of strong contrast or distinct patterns, enabling reliable matching across different images. \\

    \textbf{Local Matching} & 
    Matching keypoints between images by comparing descriptors, focusing on individual feature descriptors to establish precise localization. \\

    \textbf{Mutual Information} & 
    A metric quantifying the information shared between two images, aiding in assessing the quality of feature matching and overlap similarity. \\

    \textbf{Planar Transforms} & 
    General transformations applied to a plane in 2D space, such as affine transformations, homography, scaling, and shearing, which are crucial for image alignment. \\

    \textbf{Scaling} & 
    A transformation that changes the size of an image or its features without altering its shape, normalizing feature sizes across different images. \\

    \textbf{Shearing} & 
    A type of affine transformation that slants the shape of an object in an image. Shearing changes angles between lines while keeping parallel lines intact. \\

    \textbf{Localization} & 
    The process of determining the position of an object or feature within a given space, used in the context of GPS and image-based navigation. \\

\end{tabularx}
\endgroup

\newpage
\subsubsection*{Acronyms and Abbreviations}

\begingroup
\renewcommand{\arraystretch}{1.2}
\begin{tabular}{@{}p{2.5cm} l}
    GPS     & Global Positioning System \\
    RMSE    & Root Mean Square Error \\
    MAE     & Mean Absolute Error \\
    MI      & Mutual Information \\
    UAV     & Unmanned Aerial Vehicle \\
    GNSS    & Global Navigation Satellite System \\
    DOF     & Degrees of Freedom \\
\end{tabular}
\endgroup
