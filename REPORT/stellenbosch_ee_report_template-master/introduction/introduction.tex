\graphicspath{{introduction/fig/}}

\chapter{Introduction}
\label{chap:introduction}




\section{Background}
Unmanned Aerial Vehicles (UAVs) have become critical in modern military operations, providing capabilities in surveillance, reconnaissance, and intelligence gathering. In South Africa, UAVs are increasingly used to monitor borders and support military missions by enabling persistent aerial observation \cite{Weiss2024}. Their ability to operate in dangerous or inaccessible areas makes them invaluable for various operations.

However, UAVs heavily rely on the Global Positioning System (GPS) for navigation, which poses a significant vulnerability. GPS signals are easily disrupted by jamming, spoofing, or interference in hostile environments, leading to navigation failure. Adversaries increasingly exploit these vulnerabilities, resulting in UAVs becoming disoriented or lost, jeopardizing both the mission and the UAV. 

The increasing prevalence of GPS disruptions highlights a critical need for alternative navigation solutions that can operate independently of external signals like GPS. These systems must be robust, accurate, and adaptable to a wide range of environments. Some alternatives include inertial navigation systems, radio frequency-based localization, and quantum positioning systems. However, these suffer from high costs, limited accuracy, susceptibility to detection, Earth curvature issues, or drift, making them impractical for widespread UAV deployment. One promising approach that does not fall prey to these issues is image-based navigation, where UAVs use onboard cameras to capture and match images with previously stored reference images to estimate their position. This method allows for autonomous navigation in GPS-denied environments without requiring continuous GPS access.

However, image-based systems face uncertainty regarding their ability to offer real-time operation, achieve accurate results, and handle various environments. This project aims to test the viability of such a system, addressing the need for reliable UAV navigation without GPS dependency.

While conditions such as poor lighting and dynamic environments can pose practical challenges, advancements in technology and image processing algorithms have significantly improved the robustness of computer vision tasks to these challenges.

By leveraging these advancements, this project will develop and evaluate a system that employs the latest image processing techniques to deliver accurate and real-time localization for UAVs in GPS-denied environments. The solution will be rigorously tested against stringent performance metrics to assess its practicality and reliability in real-world scenarios.





\section{Problem Statement}
GPS-based UAV navigation is increasingly vulnerable to jamming and spoofing, risking loss of control. Current alternatives are either too expensive or unreliable, creating a need for a more robust, GPS-independent solution.

\section{Aims and Objectives}
\subsection{Aim}
The primary aim of this project is to develop an accurate and generalizable image-based GPS localization system to facilitate UAV navigation in GPS-denied environments.

\subsection{Objectives}
To achieve the stated aim, the following objectives have been established:

\begin{enumerate}
    \item \textbf{Optimize the Localization Pipeline:} Identify and implement the most effective pipeline, including parameter selection and methodological approaches, to achieve the highest possible localization accuracy while maintaining real-time performance and generalizability. This includes the selection, integration and optimization of feature extraction, matching, and planar transformation estimation techniques.
    \item \textbf{Accurate Localization Estimation:} Estimate the GPS location, using only prior telemetry data and image features, with a radial error below 10\% of the UAV's radial displacement from the reference image. This accuracy is sufficient for manual control of the UAV in GPS-denied environments, given that the system provides continuous correction.
    \item \textbf{Real-Time Operation:} Ensure the localization system operates in real-time, with a response time of less than 2 seconds following GPS signal loss. This is sufficient as the relative motion of the UAV to the ground is sufficiently slow that a 2-second delay will not significantly affect the pilots ability to navigate.
    \item \textbf{Environmental Generalizability:} Validate that the system maintains its performance metrics across diverse environments without the need for environment-specific parameter tuning.
\end{enumerate}

\section{Summary of Work}
This project successfully developed and implemented an image-based GPS localization system for UAV navigation in GPS-denied environments. The system met the outlined objectives, demonstrating high accuracy, real-time performance, and generalizability across multiple environments. Through comprehensive research and implementation of feature extraction, matching, and homographic techniques, the system effectively estimated the UAV's location with minimal error. The system was tested and deemed to be practically robust under low-light and limited overlap conditions, with the ability to handle various environments. The outcomes indicate significant potential for enhancing UAV navigation reliability in both military and civilian applications.

\section{Scope}
\label{sec:scope}
The scope of this project is defined to ensure feasibility and manageability within the allocated timeframe. The limitations are considered reasonable in practical scenarios or can be addressed in future, more detailed studies. These constraints do not hinder the ability to assess the solution's viability. The following assumptions and restrictions outline the project's boundaries:

\begin{itemize}
    \item \textbf{No Perspective Distortion:} The UAV's downward-facing camera is assumed to remain perfectly aligned downward throughout all operations, so no pitch or roll occurs. Additionally, the ground is assumed to be predominantly planar (2-dimensional), simplifying the model by removing the need for perspective distortion correction.
    
    \item \textbf{No Scale Distortion:} The UAV maintains a constant altitude, negating the requirement for scale estimation in localization calculations.
    
    \item \textbf{Sufficient Image Quality:} It is assumed that the captured images have adequate resolution and full visibility without significant distortion.
    
    \item \textbf{No Dynamic Movements:} The UAV is expected to operate over primarily static land surfaces with minimal dynamic objects, ensuring a stable environment for image-based localization.
    
    \item \textbf{Path Overlap:} The UAV leverages complementary odometry-based systems to determine its path. This system assumes that the UAV has located the path, and the overlap of reference image information is greater than 60\%.
    
    \item \textbf{Method and Parameter Set Testing:} The study aims to assess the system's viability given sufficiently accurate methods and parameter sets, focusing on the overall performance of the system rather than on exhaustive testing of optimal methods or fine-tuning of parameters.
    
    \item \textbf{Training Requirements:} Machine-learning-based methods are excluded due to the extensive training data required, which is outside the project's scope.
    
    \item \textbf{Data Provisioning:} The project used data from Google Earth due to not receiving the prerequisite data, acknowledging a minor accuracy impact due to the use of estimated ground truth heading data.
    
    \item \textbf{System Output:} The system provides an estimated GPS location and heading, intended for manual UAV control by pilots rather than for automated control systems.
    
\end{itemize}

These scope limitations streamline the project while allowing for future enhancements, such as incorporating scale and perspective correction or improving robustness to dynamic environmental conditions.

\section{Structure of the Report}
This report is structured to provide a comprehensive overview of the research undertaken to develop the image-based GPS localization system for UAVs. The chapters are organized as follows:

\begin{itemize}
    \item \textbf{Chapter 2: Literature Review} \\
    This chapter reviews the current state of GPS-alternative UAV navigation systems, emphasizing their drawbacks and associated vulnerabilities. It provides the reader with the prerequisite knowledge to understand the techniques employed in this study, including feature detection and matching techniques, and planar transformation estimation methodologies. 
    
    \item \textbf{Chapter 3: Methodology} \\
    This chapter details the system pipeline designed to achieve accurate, efficient, and robust image-based GPS localization. It covers the flow and integration of feature extraction, matching, and homographic estimation, as well as the associated considerations made to optimize the system's performance.
    
    \item \textbf{Chapter 4: Testing} \\
    This chapter describes the datasets utilized, the testing environment setup, and the comparative analysis of different methods employed at each stage of the pipeline. It culminates in the selection of the final methods integrated into the system.
    
    \item \textbf{Chapter 5: Results} \\
    This chapter presents the evaluation of the developed pipeline against the established objectives. It includes stress testing results and assesses the system's performance in various scenarios to demonstrate its effectiveness and reliability.
    
    \item \textbf{Chapter 6: Conclusion} \\
    The final chapter summarizes the project's outcomes, and evaluates the viability of the developed system.
    
    \item \textbf{Chapter 7: Future Work} \\
    This section outlines recommendations for future work to address the project's limitations and explore additional enhancements.
\end{itemize}














