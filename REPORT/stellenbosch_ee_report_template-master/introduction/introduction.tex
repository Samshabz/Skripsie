\graphicspath{{introduction/fig/}}

\chapter{Introduction}
\label{chap:introduction}


CH1
 Background (1/2 to 1 page) - give context
- background is not what is the problem, but what is the context of the problem. Why does it matter. How many people suffer from lack of clean water for eg or die from car accidents. 

- problem statement (1 paragraph). Most important of entire report. Exactly what you have done. Nothing more or less. sentence or two. 
- objectives "fine-grain" SMART, numbers. If i meet these objectives, i have solved the problem statement in part. 1-5 objectives or sub-objectives or even requirements. Measured against objectives - metrics. 
Summary of work (not contribution for skripsie). what you have achieved. meeting objectives. 
- Scope: What will be done and not be done. Mainly what will not be done. Did not take into account organic material or eg did not consider mobility. Did think about other things - still important (e.g. future work). write it in such a way that it is positive  / what u did do. did not consider cost size market manufacturability etc. 
- Structure of Report (Not exactly the same as the Table of Contents): Roadmap, in chapter 2 I did the following etc

ECSA chapter 1: I took a vaguely defined problem (prob statement), understood what i must do to solve (objectives), and then solved it (summary of work).
Then, 




\section{Introduction}
\begin{itemize}
    \item \textbf{Background:}
    \begin{itemize}
        \item To maintain state safety, Unmanned Aerial Vehicles (UAVs) are used in military missions for surveillance, reconnaissance, and intelligence gathering. However, they rely on GPS for navigation, which can be jammed, spoofed, or otherwise lost in adversarial environments. This severely limits the scope of where these missions can be conducted, as well as the safety of the UAVs.
    \end{itemize}
    \item \textbf{Problem Statement:}
    \begin{itemize}
        \item Traditional UAV navigation systems are GPS-based, which is susceptible to new GPS-denial technology. Additionally, existing solutions mainly involve emission technology (e.g., LIDAR) which can be detected by adversaries.
    \end{itemize}
    \item \textbf{Motivation:!!!}
    \begin{itemize}
        \item There is a need for a navigation redundancy system that avoids detection in GPS-denied environments. Intertial reference units drift over time FIX LATER. weighted based on confidence level flip. 
    \end{itemize}
\end{itemize}

\section{Aims and Objectives}
\begin{itemize}
    \item \textbf{Aim:}
    \begin{itemize}
        \item Develop a robust and accurate image-based GPS location estimator for UAV navigation post-GPS signal loss. 
    \end{itemize}
    \item \textbf{Objectives:}
    \begin{itemize}
        \item Design, develop and test algorithms for efficient, robust and accurate feature extraction. The system shall have a good feature count of at least 1000 per image.
        \item Design, develop and test algorithms for efficient, robust and accurate feature matching. The system requires at least 500 good matches to be sufficiently confident. 
        \item Use the matching algorithms to infer the changes in translation and rotation between the current and prior images when GPS signal is lost. This ultimately allows for the estimation of the UAV's new GPS location and heading. The system shall have a normalized x-y error under 100m and a heading error under 10 degrees for all estimations.
        \item Improve the above method by utilizing stereo vision for depth awareness and increased accuracy. The system shall have a less than half the error of the monocular system.
        \item Implement the system on a single-board computer (SBC) for real-time operation. The system shall have a processing time of under 10s per frame in the forward flight, and under 3s per frame in the backward flight.
    \end{itemize}
\end{itemize}

\section{Scope}
\begin{itemize}
    \item \textbf{Assumptions and Restrictions:}
    \begin{itemize}
        \item Real-life footage from a UAV test-flight will be used. This camera shall be aimed perfectly downwards for all time, sufficiently detailed and distortion-free (e.g., warping, blurring, clouds, etc.).
        \item The UAV will be flying at a constant altitude. 
        \item There will therefore be limited depth warping. That is, the accuracy of the system using 2D estimation will be sufficiently accurate`' but there will still be some room for improvement in 3D.
        \item The UAV and its cameras will always be level with the ground. That is, no pitch and roll will occur. 
        \item The system output is an estimated GPS location and heading estimate, not a control system. This is sufficient for the navigators to manually control the UAV.
    \end{itemize}
\end{itemize}

\section{Methods and Approach}
\begin{itemize}
    \item \textbf{Data Collection:}
    \begin{itemize}
        \item Capture sequential images from UAV flights at different altitudes.
        \item The UAV should follow a predefined path with known GPS coordinates, heading, camera parameters, and altitude. Thereafter, the UAV should fly very close to the same path backward.
    \end{itemize}
    \item \textbf{Design and Development:}
    \begin{itemize}
        \item Research and test the accuracy and count of features in 5-10 different feature extraction methods. 
        \item Research and test the accuracy and count of matches in 5-10 different feature matching methods.
        \item Implement and evaluate different homographic estimation methods. 
        \item Implement and evaluate stereo vision methods.
        \item Implement and evaluate the system on a SBC.
    \end{itemize}
    \item \textbf{Analysis:}
    \begin{itemize}
        \item Use sequential image analysis to estimate new coordinates.
        \item Implement and evaluate pathfinding algorithms to calculate a path backward using past captured features.
    \end{itemize}
\end{itemize}


\section{Summary of Work}
\begin{itemize}
    \item The project aims to develop a robust and accurate image-based GPS location estimator for UAV navigation post-GPS signal loss. The system will be designed, developed, and tested for feature extraction, matching, homographic estimation, stereo vision, and computational requirements. The system will be implemented on a SBC for real-time operation.
    \item The project will have a significant impact on UAV navigation accuracy and reliability in GPS-denied environments, with potential applications in military and civilian contexts.
    
\end{itemize}


\section{Structure}

\begin{itemize}
    \item \textbf{Chapter 2: Literature Review}
    \begin{itemize}
        \item Discusses the current state of UAV navigation and GPS dependency, alternative navigation methods, feature detection and matching, homographic estimation, stereo vision, and computational requirements.
    \end{itemize}
    \item \textbf{Chapter 3: Methodology}
    \begin{itemize}
        \item Describes the data collection, design and development, and analysis methods used in the project.
    \end{itemize}
    \item \textbf{Chapter 4: Results}
    \begin{itemize}
        \item Presents the results of the project, including the accuracy and efficiency of the developed system.
    \end{itemize}
    \item \textbf{Chapter 5: Discussion}
    \begin{itemize}
        \item Discusses the implications of the results and the potential impact of the project.
    \end{itemize}
    \item \textbf{Chapter 6: Conclusion}
    \begin{itemize}
        \item Summarizes the project and its outcomes.
    \end{itemize}
\end{itemize}



\section{Potential Impact}
\begin{itemize}
    \item \textbf{Societal Impact:}
    \begin{itemize}
        \item Enhance the reliability of UAV operations in civilian applications such as disaster response and border patrol.
    \end{itemize}
    \item \textbf{Knowledge Impact:}
    \begin{itemize}
        \item Contribute to advancements in UAV navigation technology and computer vision applications.
    \end{itemize}
    \item \textbf{Economic Impact:}
    \begin{itemize}
        \item Increase the practicality and usage of UAV systems in military and civilian contexts, potentially enhancing border control, enhancing international electronic engineering collaboration.
    \end{itemize}
    \item \textbf{Safety Impact:}
    \begin{itemize}
        \item Increase the ability for national surveillance missions to secure critical knowledge in a military context.
    \end{itemize}
\end{itemize}

\section{Conclusion}
\begin{itemize}
    \item The project aims to improve UAV navigation accuracy and reliability in GPS-denied environments through camera-based image matching. This may have profound impacts in a South African military context as well as broader impacts on civil safety and economic well-being.
\end{itemize}

\section*{References}
\begin{enumerate}
    \item Sathyamoorthy, D., et al. (2020). Evaluation of the Vulnerabilities of Unmanned Aerial Vehicles (UAVs) to Global Positioning System (GPS) Jamming and Spoofing. \textit{Defense Science \& Technology Journal}, 33, 334-343.
    \item Rusnák, M. \& Vásárhelyi, J. (2023). A review of using visual odometry methods in autonomous UAV navigation in GPS-denied environment. \textit{Acta Universitatis Sapientiae Electrical and Mechanical Engineering}, 15, 14-32. \url{https://doi.org/10.2478/auseme-2023-0002}
    \item Karab, E., Prasad, S., \& Shahato, M. (2017). Image Matching Using SIFT, SURF, BRIEF and ORB: Performance Comparison for Distorted Images. \textit{Acta Polytechnica Hungarica}, 14(6), 183-206.
    \item Luglio-Miguel, F., Iores, G., \& Salazar, S. \& Lozanc, R. (2014). Dubins path generation for a fixed-wing UAV. \textit{Proceedings of the International Conference on Unmanned Aircraft Systems (ICUAS)}, 339-345.
\end{enumerate}
