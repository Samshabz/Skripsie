\graphicspath{{conclusion/fig/}}

\chapter{Summary and Conclusion}
\label{chap:conclusion}

\vspace{-1cm}


GNSS vulnerabilities, such as jamming and spoofing, present significant risks to Unmanned Aerial Vehicle (UAV) navigation, potentially resulting in loss of control and mission failure. Recognizing the limitations of existing alternatives—often prohibitively costly or unreliable—this project aimed to address the urgent need for a robust, GNSS-independent navigation system for UAVs operating in GNSS-denied environments. The primary objective was to develop an accurate, adaptable, and generalizable image-based navigation system as a redundancy to GNSS. This objective was pursued through the careful selection and integration of effective feature extraction, matching, similarity computation, and planar transformation estimation techniques to optimize the overall system pipeline.

The resulting implementation met all outlined objectives. It achieved a mean accuracy of under 2\%, significantly surpassing the initial target of 10\%. The system maintained real-time performance during the GNSS available phase-where it captured features in the outbound path-and during GNSS-denied scenarios-where it outputted its with response times consistently under 2 seconds across all datasets.

Despite challenges such as terrain variability, quantization errors, depth changes, and optical distortions from camera lenses, the system upheld robust accuracy across diverse datasets. Its resilience was further proven under varying lighting conditions, static camera tilt, reference image overlap, and resolution changes, showcasing its adaptability to real-world operational demands. However, it is important to acknowledge that the system’s performance was notably impacted under extreme low-light conditions.

In conclusion, this project successfully developed a versatile and effective image-based GNSS redundancy navigation system for UAVs, demonstrating consistent and reliable performance across a wide range of conditions. With future enhancements, this system holds significant potential for strengthening UAV navigation reliability, enhancing mission safety, and supporting critical operations in national security and beyond.



\section{Recommendations for Future Work}

This study establishes a foundational framework for image-based UAV navigation, with several avenues for enhancement. Future work could incorporate a multi-image weighted inference system to improve location accuracy by leveraging multiple reference images, thereby mitigating individual image distortions or outliers. Additionally, implementing non-planar stereo matching would enhance depth perception and altitude estimation in complex terrains. Integrating up-to-date mapping data with reference images could eliminate the need for prior image capture and fixed return paths, allowing UAVs to navigate freely within mapped regions without positional drift. Addressing distortions from roll, pitch, and altitude changes through homography transformations would further enhance accuracy. Moreover, deploying the system on higher-performance Single Board Computers (SBCs) and utilizing higher-resolution imaging can support more sophisticated computations and detailed feature extraction, respectively. These advancements would collectively ensure a more robust, adaptable, and scalable navigation system suitable for diverse UAV applications.


