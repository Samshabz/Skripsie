The next critical feature in an image-based navigation system is the feature matcher. The feature matcher serves two main purposes. The first is to find the most highly correlated image pair. The second is to estimate the relative pose (translation and rotation) between the two images. Thus, we break up this section into two components, global matching and local matching. 
xxx check scale if or not if
To compute matches, The descriptor first transforms each feature individually to a normalized space. This process aligns the feature’s orientation, adjusts scale, and normalizes other factors like intensity, ensuring that each descriptor is in a consistent format for direct comparison during matching.


Matches have specific confidence levels. When calculating transformations we can include confidence weightings to improve the accuracy of the estimations. 

Broadly speaking there are two types of feature matchers: local and global. Local matchers are used to find the relative pose between two images. Global matchers are used to find the correlation between two images.

\section*{Local Matchers}
\subsection*{Brute-Force Matcher (BFMatcher)} BFMatcher compares every descriptor from one image with all descriptors in another image to find the closest match. It calculates the distance between descriptors using a chosen metric (e.g., Euclidean for SIFT, Hamming for ORB). Matches are sorted by distance, and the best ones are selected. While simple and effective, it can be slow with large datasets. Best suited for smaller datasets or when precision is prioritized over speed.

\subsection*{FLANN (Fast Library for Approximate Nearest Neighbors)} FLANN is an efficient matcher that uses approximate nearest neighbor search, speeding up the matching process, especially for large datasets. It uses tree-based algorithms or k-means clustering to quickly find matches. FLANN is often preferred for tasks involving large datasets where speed is critical, such as real-time applications with SIFT or SURF descriptors. FLANN must be used with Local-sensitivity hashing to allow for operations on its binary descriptors. 



\subsection*{Light Glue} Light Glue is a lightweight, relative to other neural network approaches, neural network-based matcher that uses deep learning to match features across images. It is optimized for speed and memory efficiency, making it suitable for mobile and embedded systems. Light Glue provides robust matches even under challenging conditions, such as significant viewpoint changes or lighting variations. This is a significant improvement over traditional feature matching methods like BF or FLANN matchers, however, it increases computational time and risks an inability to generalize well, especially in this context which is unlikely to have training data based off UAV images.

\section*{Improvements to Matching techniques}
\subsection*{Cross-Check Matching} Cross-check matching performs matching in both directions (from image A to B and B to A) and only retains matches that are consistent in both directions. This reduces false positives and increases the reliability of matches. However, this doubles the computation time.
 

\subsection*{RANSAC (Random Sample Consensus)} RANSAC is used to refine matches by identifying and removing outliers. It works by iteratively selecting a subset of matches, estimating a model (e.g., homography), and checking how well the remaining matches fit this model. Matches that deviate significantly are considered outliers and discarded. RANSAC is crucial in tasks like image stitching, where accurate geometric transformations are required.


\subsection*{SuperGlue} SuperGlue is a more advanced neural network-based matcher that leverages attention mechanisms, dynamically aggregating local features based on their inferred importance, to enhance the matching process. It works by learning to match keypoints directly from image data, providing superior performance in complex scenarios with large changes in scale, rotation, or perspective. SuperGlue is well-suited for high-precision applications like 3D reconstruction. 


\section*{Global Matchers}

Global matchers aim to capture the overall similarity between images. Many techniques, like local matchers, do not inherently test similarity evenly across the image and therefore may tend to compare local zones instead of the entire context. This leads to sub-par matching and pose inference. 
To ensure full context, we can either choose a global matcher that already considers the entire image space, or use one that does not and divide into grids and compare per-grid to ensure complete context. In both cases, the scale and rotation should be normalized between images unless explicitly invariant to these. 
The following are chosen based on their ability to be computationally efficient and not require any pre- or live-training.

\subsection*{Translation affects the results of these}

\subsection*{Local matching conversion techniques}
These are techniques which convert local matchers, which are translation invariant, into techniques which penalize translation in its score. From the above local matchers, tests will be conducted on 

\subsection*{Histograms}

\subsection*{Hashing Methods}  
Hashing methods, like LSH (Local-sensitivity hashing) or PCA (Principal-component analysis), map high-dimensional descriptors to a lower-dimensional space. Hashing focuses on descriptor similarity rather than relative spatial positions within the image. Grid-rot apply

\subsection*{SSIM (Structural Similarity Index)}  
SSIM compares images based on luminance, contrast, and structural information, making it sensitive to spatial shifts like translation. 

\subsection*{KNN or other local Matching}  
Adapting local matchers, such as AKAZE, for global matching allows for high accuracy. 

Use low threshold, per-grid matching, and count the amount of matches. This is point-to-point and can be computationally expensive when comparing many images. 



\subsubsection*{Plan}
These methods offer complex tradeoffs within different accuracy and speed metrics. As such, tests will be conducted on individual methods as well as hybrid approaches. All of the above methods will be included in these tests except the deep-learning approaches which are too computationally intensive for the task of finding the most correlated image. The best method is that which results in the combination of image pairs (matches) which subtends the lowest mean squared deviation in actual and estimated GPS locations and heading. 


To summarize: the invariant methods are: Hashing, BOVW; variant methods are 



Intermediary results:

These results indicate the time to complete the set of best correlated images. The score has not been used as the highly variant parameter set often, with enough computation, can achieve the correct score. So instead of holding computational time constant, which is not clear, we increase the parameters until we see it achieve the correct score. Then the time is recorded and a higher accuracy is dually implied in a lower time. Where excessive time is required and the correct combination has not been met, there will be an indication of failure to meet score. 
Excessive time is defined as greater than 10s for an image space of 12. 

\begin{table}[H]
    \centering
    \begin{tabular}{|c|c|}
    \hline
    \textbf{Search Space (Images)} & \textbf{Runtime (Seconds)} \\ \hline
    12  & 5.04 \\ \hline
    11  & 5.14 \\ \hline
    10  & 4.90 \\ \hline
    9   & 4.30 \\ \hline
    8   & 3.74 \\ \hline
    7   & 3.56 \\ \hline
    6   & 3.12 \\ \hline
    5   & 2.59 \\ \hline
    4   & 1.99 \\ \hline
    3   & 1.51 \\ \hline
    2   & 0.98 \\ \hline
    1   & 0.49 \\ \hline
    \end{tabular}
    \caption{KNN Matching: Search Space vs. Runtime}
    \end{table}

For BF_matching with KNN search, the run-time per image is roughly half a second, and the time complexity is linear.   
