\textbf{Note:} The runtime of individual methods should not be used for direct time comparisons, as it must be evaluated within the context of the entire pipeline. For example, a match filtration method might execute faster because it filters out fewer matches, but the resulting increase in matches could slow down subsequent steps. The goal is to assess the overall impact of the method on the entire pipeline, not just the method’s isolated runtime.

THEORY To Be Completed. 


\section{Rotational Estimators}

\subsection{Initial Accuracy-Runtime Testing} Initial tests were conducted with optimal, yet generalizable, parameters to evaluate each method's Mean Absolute Error (MAE) in heading estimation. Since ground truth heading data was unavailable for heading changes, a dataset with fixed North headings was used. Note that methods without built-in RANSAC that aimed to estimate purely based on the point cloud proved ineffective in empirical testing.
Results are summarized in Table \ref{tab:initial_rotation_testing}. 


\begin{table}[H]
    \centering
    \caption{RMSE Comparison Across Datasets for Partial Affine 2D, Affine 2D, and Homography}
    \label{tab:rmse_comparison}
    \begin{tabular}{|c|c|c|c|}
    \hline
    \textbf{Dataset} & \textbf{Partial Affine 2D} & \textbf{Affine 2D} & \textbf{Homography} \\
    \hline
    CITY1   & 70.18 & 81.22 & 76.25 \\
    CITYT2  & 7.54  & 7.80  & 7.16  \\
    ROCKY   & 27.98 & 22.10 & 24.40 \\
    DESERT  & 44.94 & 43.11 & 48.56 \\
    AMAZON  & 46.98 & 52.56 & 48.04 \\
    \hline
    \end{tabular}
    \end{table}
    
    
    \begin{table}[H]
        \centering
        \caption{Runtime Comparison Across Datasets for Partial Affine 2D, Affine 2D, and Homography}
        \label{tab:runtime_comparison}
        \begin{tabular}{|c|c|c|c|}
        \hline
        \textbf{Dataset} & \textbf{Partial Affine 2D} & \textbf{Affine 2D} & \textbf{Homography} \\
        \hline
        CITY1   & 56.76 & 65.27 & 68.79 \\
        CITYT2  & 52.20 & 54.55 & 65.11 \\
        ROCKY   & 69.44 & 76.16 & 78.92 \\
        DESERT  & 56.40 & 71.52 & 69.97 \\
        AMAZON  & 50.34 & 58.16 & 58.27 \\
        \hline
        \end{tabular}
        \end{table}
        






\begin{table}[H]
    \centering
    \begin{tabular}{|c|c|c|}
        \hline
        \textbf{Method} & \textbf{MAE heading (degrees)} & \textbf{Result} \\
        \hline
        Homography & 0.1207 & Accurate \\  
        Affine & 0.1129 & Best-performing \\  
        2x2 Rotation Matrix & 1.0277 & Poor performance \\  
        Vector-based & 7.3571 & Poor performance \\  
        \hline
    \end{tabular}
    \caption{Initial Testing Results (MAE - Mean Absolute GPS Error)}
    \label{tab:initial_rotation_testing}
\end{table}

\textbf{Conclusion:} The affine and homography methods performed sufficiently well to be considered for further testing. The 2x2 rotation matrix and vector-based methods were discarded due to poor performance. Their inability to iteratively solve and filter matches led to significant error points, making them unsuitable for the task at hand.


\subsection{Detailed Time, Robustness \& Accuracy Testing}


\begin{table}[H]
    \centering
    \begin{tabular}{|c|c|c|c|c|}
    \hline
    \textbf{Dataset}   & \multicolumn{2}{c|}{\textbf{Homography}} & \multicolumn{2}{c|}{\textbf{Affine}} \\ \hline
    & \textbf{RMSE GPS} & \textbf{Runtime (s)} & \textbf{RMSE GPS} & \textbf{Runtime (s)} \\ \hline
    CITY1 & 73.7828                         & 66.1684           & 65.5642            & 62.3034           \\ \hline
    CITY2 & 7.5038                          & 68.6799           & 7.3890             & 64.9879           \\ \hline
    ROCKY & 23.2824                         & 88.6713           & 22.3195            & 80.8048           \\ \hline
    DESERT & 49.0210                        & 76.0751           & 38.6201            & 72.3996           \\ \hline
    AMAZON & 48.6687                        & 69.6631           & 44.5493            & 68.0990           \\ \hline
    \end{tabular}
\end{table}




\textbf{Observations:} Affine consistently outperformed homography in both accuracy (RMSE GPS) and runtime across all datasets. Homography's additional degrees of freedom (namely, perspective transformation) introduce unnecessary complexity and error when such transformations aren't significantly present.

\subsection{Method Sensitivity} Sensitivity tests were conducted to assess each method’s robustness to parameter variations (namely, Lowe's ratio, RANSAC thresholds, detector type, keypoint quantity). Both methods exhibited robustness to parameter changes, though homography showed marginally higher deviations when only having access to few matches, due to its more complex model. 
Both methods rated a 5/5 for robustness across varying datasets and conditions.

\subsection{Conclusion} Affine estimation is selected for future work due to its superior accuracy, computational efficiency, and robustness. With its 6 degrees of freedom, it strikes a good balance for this task. However, if the UAV flies at lower altitudes where perspective distortion becomes significant, the homography method may need to be reconsidered.






insert image showing generalized rot pattern which angles. ie how to rotate. 