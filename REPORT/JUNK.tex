ROT ESTIMATION STARTSTART

Methods tried:
HOMOGRAPHY ROTATIONAL CALCULATERS




IMPROVEMENT TECHNIQUES
LOWES with KNN
RANSAC - very stable, high accuracy; uses lowes too
Confidence threshold - stability issues
Confidence weightings - stability issues, aiming to IMPROVEMENT

The latter two are not used because:
Upon testing various match confidence weightings and thresholds, it was found that they tend to decrease accuracy and stability (deviation). That is, they are less reliable and of a lower mean squared error. The reason for this is the already robust, outlier removal techniques used such as Lowes Ratio, and RANSAC with homography. That means that decreasing the amount of matches further is not only redundant but also detrimental to the accuracy of the system. Using these techniques without Lowes Ratio and RANSAC also proved to be less accurate and stable.


INITIAL TESTS ACC1-----------
EXPANDED MATCH ROTATION METHOD - ALL (ONLY GOOD) potential image MATCHes TESTed. 
method 1: cv2.findHomographyL MAE HEADINGS: 0.12066861930614703
method 3: cv2.estimateAffine2D MAE HEADINGS: 0.11285236301448347 
METHOD 2: 2x2 Rotation Matrix with partial affine MAE HEADINGS: 1.0277104256025862 NOT NUANCED ENOUGH. AffinePartial only has 4 degs freedom, rot and translation. 
METHOD 3: Vector-based MAE HEADINGS: 37.35710735665052, NOT NUANCED ENOUGH
---------------------


ACCURACY 2--------
METHOD0-0: homography - homography

Preprocessing Global Detector: AKAZE, Preprocessing Global Matcher: BF, Global Matching Technique: Histogram, Local Detector: ORB, Local Matcher: GRAPH
48.89603659984369

METHOD2-2: affine  - affine
Preprocessing Global Detector: AKAZE, Preprocessing Global Matcher: BF, Global Matching Technique: Cross Correlation, Local Detector: ORB, Local Matcher: GRAPH
Mean normalized error: 41.281462286369305


This test is to see if there is a bias towards the responsivity of rotational errors to global or local matchinf techniques.

METHOD 0(GLOBAL) - 2(LOCAL): homography -> affine
Mean normalized error: 41.80079808674839



METHOD 2(GLOBAL) - 0(LOCAL): affine -> homography   
Mean normalized error: 42.36514574207946


Global uses rotational alignment to then infer the most similar image. Local uses rotational alignment to infer the translational shift. As visible, affine works better in both cases. However, whats important to note is that there is a bigger change to accuracy when changing the local method than the global method. This is because the local method is more sensitive to the rotational error, or more rotationally variant, than the global method. In future, we need to aim to reduce that dependency by researching other methods of translational estimation that are less rotationally variant. 
-----------

Robustness, TIME and Accuracy 3 -------
-To params 
Robustness measure: GIVEN random parameters for : 
Homog uses: homography\_threshold 
Affine uses: ransacReprojThreshold
This is useful because new datasets might have vastly different depths and environments. This is a good way to ensure that the system is robust to these changes.

for same keypoints and partly optimal params, the mean runtime per iteration of the method line is: 
- THIS Can be used for time analysis of methods
Affine:
Mean Time: 1.275981797112359 , Max Time: 11.546134948730469, Min Time: 0.0 ms, Med time is: 0.9990930557250977

HOMOG:
Mean Time: 22.58008321126302 , Max Time: 50.74262619018555, Min Time: 0.9782314300537109 ms, Med time is:  median: 10.803937911987305

As seen above, both methods have spread due to difference in the amount of keypoints to analyse. However, the difference between mean and median is especially visible in the homography method. This indicates that there are significant spikes. Thus, the stability of the homography method is significantly lower. 


$$method_using: 0$$
$$
VERBATIM
AFFINE verbatim: 
    M, mask = cv2.estimateAffine2D(src_pts_estimate, dst\_pts_estimate , method=cv2.RANSAC, ransacReprojThreshold=0.5)
HOMOG verbatim
    M, mask = cv2.findHomography(src_pts, dst_pts, cv2.RANSAC, homography_threshold)
    $$
 
POOR PARAM CHOICE ROBUSTNESS TEST - direct param changing thresholds
    Preprocessing Global Detector: AKAZE, Preprocessing Global Matcher: BF, Global Matching Technique: Cross Correlation, Local Detector: AKAZE, Local Matcher: BF    - This is what Im basically running all the tests in this doc on. So maybe say that initially. 
    A higher homography and ransac threshold allows more keypoints. 
- This should be used for Robustness to param choices. 
\#METHOD 0: homography, default 0,5 or 25 ORB 
TEST1\_param = 0.05 : Mean Heading Error: 0.148998095687098
TEST2\_param = 0.2 : Mean Heading Error: 0.13841201314473225
TEST2\_param = 0.5 : Mean Heading Error: 0.12066861930614703 DEFAULT LOCMIN
TEST3\_param = 5 : Mean Heading Error: 0.12498785307973244
TEST4\_param = 25 : Mean Heading Error: 0.12218131739734368
TEST4\_param = 50 : Mean Heading Error: 0.12259957394613104
ORB TEST: 



\#METHOD 1: affine, default 0.5
TEST1\_param = 0.05 : Mean Heading Error: FAIL - INSUFFICIENT PTS
TEST2\_param = 0.2 : Mean Heading Error: 0.1089152618821029
TEST2\_param = 0.5 : Mean Heading Error: 0.11285236301448347 DEFAULT
TEST3\_param = 5 : Mean Heading Error: 0.10034144212143525
Test4\_param = 25 : Mean Heading Error: 0.0997489192566366 -> NEW DEFAULT LOCMIN AKAZE 
TEST4\_param = 50 : Mean Heading Error: 0.11119680707008499

- Reasoning for consistency:
AFFINE: Supports: rotation, scaling, translation, and shearing.
HOMOGRAPHY: Supports: rotation, scaling, translation, shearing, and perspective.

From the above results, it is evident that, homographic estimation, with its 8 degrees of freedom results in a more comprehensive and stable understanding of images. It has more consistent results and does not result in over-filtering when using strict thresholds. However, it is also evident that the affine method is more accurate and faster. The chief reasoning for the latter would be the introduction of perspective matching when using homography which allows for extra error points to be introduced when it is not necessary. 


- Outlier removal techniques param changes (LOWES) to see robustness

TRYING WITH LOWES CHANGES with BF KNN
homog:
0.6: Mean Heading Error: 0.12885225757810576
0.7: 0.12548004268378335    
0.8: 0.12066861930614703
0.9: 0.11388668083394694 LOCMIN
0.95: 0.12105176176293807

Preprocessing Global Detector: AKAZE, Preprocessing Global Matcher: BF, Global Matching Technique: Cross Correlation, Local Detector: AKAZE, Local Matcher: BF
Affine:
0.6: 0.10975705804226184
0.7:0.10060366005984626
0.8: 0.0997489192566366 LOCMIN
0.9: 0.11105935025493115
0.95: 0.10538102537306442

The above is accuracy of keypoints vs number of keypoints (via matches) - and how it affects results. 


- Robustness to the amount of keypoints 
AFFINE
0.001:  0.1019014570444459
0.0008: 0.09735398402697985 -  a higher threshold implyies less keypoints. faster too. LOCMIN
0.0005: 0.0997489192566366
0.0002: 0.10244993743482826


HOMOG
0.0008 Mean Heading Error: 0.1403906867378703
0.0005: Mean Heading Error: 0.12066861930614703 - LOCMIN -> needs abit more points due to more degs of freedom. 
0.0002 Mean Heading Error:  0.12466228801134822 (174sec runtime)



46.44097668016442 from 46.990899180771386 with old reprojthreshold
Preprocessing Global Detector: AKAZE, Preprocessing Global Matcher: BF, Global Matching Technique: Cross Correlation, Local Detector: AKAZE, Local Matcher: BF
0.09735398402697985

Affine is decided to be used:
With optimal parameters these are the metrics: with params as per LOCMIN LINES
0.09735398402697985
Mean Rotational Normalizer Time: 0.5648016929626465 , Max Rotational Normalizer Time: 3.433704376220703, Min Rotational Normalizer Time: 0.0 ms, median: 0.2535581588745117



ROT ESTIMATION ENDEND
