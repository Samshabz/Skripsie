\documentclass[a4paper, twocolumn]{article}
\usepackage{amsmath}
\usepackage{amsfonts}
\usepackage{amssymb}
\usepackage[a4paper, margin=0.5cm]{geometry} % Adjust margins as needed

\begin{document}

\title{APPENDIX - life cycle Formulas}
\author{Sameer Shaboodien}
\date{15/08/2024}

\maketitle

The following are calculations which aim to clearly, simply, and without any layers of abstractions explain the formulas used in the excel document. Latex has been used for its superior mathematical notation. Further, in excel a note should be made that SUMIFS, IFS, and VLOOKUPS were used to sort masses and materials automatically, as can be seen in the excel document. 

\section*{Material Life Phase}

To calculate the \textbf{Material Life Phase Energy}, which accounts for the energy required for material extraction, transport, and processing, we use the following formula:

\[
E_{\text{material}} = \sum_{i=1}^{n} \left( H_{m,i} \times M_{m,i} \times Q_i \right)
\]

Where:
\begin{itemize}
    \item \( H_{m,i} \) is the Embodied Energy density (in MJ/kg) of each material \(i\).
    \item \( M_{m,i} \) is the Mass (in kg) of each material \(i\) used in the product.
    \item \( Q_i \) is the Quantity of components for each material \(i\) (number of units).
\end{itemize}

Additionally, the \textbf{Material Life Phase Carbon Footprint} is calculated as:

\[
\text{CO}_2\text{eq}_{\text{material}} = \sum_{i=1}^{n} \left( \text{CO}_2\text{eq}_i \times M_{m,i} \times Q_i \right)
\]

Where:
\begin{itemize}
    \item \( \text{CO}_2\text{eq}_i \) is the CO2 equivalent (in kg) for each kg of material \(i\).
\end{itemize}

\section*{Manufacturing Life Phase}

The \textbf{Manufacturing Life Phase Energy} considers both primary and secondary processes. The energy is calculated using:

\[
E_{\text{manufacturing}} = \sum_{i=1}^{n} \left( P_{p,i} \times M_{m,i} \times Q_i \right)\]
\[  + \sum_{j=1}^{m} \left( \text{SpE}_j \times D_j \times Q_j \right)
\]

Where:
\begin{itemize}
    \item \( P_{p,i} \) is the Primary Process Energy per unit mass for each material \(i\) (in MJ/kg).
    \item \( \text{SpE}_j \) is the Secondary Process Energy per unit dimension for each process \(j\).
    \item \( D_j \) is the dimension (such as surface area, volume, length, etc.) for each process \(j\). This is also scaled by the quantity \(Q_j\).
    \item \( M_{m,i} \) and \( Q_i \) are the Mass and Quantity for each material \(i\) as previously defined.
\end{itemize}

Similarly, the \textbf{Manufacturing Life Phase Carbon Footprint} can be calculated by:

\[
\text{CO}_2\text{eq}_{\text{manufacturing}} = \sum_{i=1}^{n} \left( P_{p,i} \times M_{m,i} \times Q_i \right)\]
\[ + \sum_{j=1}^{m} \left( \text{SpC}_j \times D_j \times Q_j \right)
\]

Where:
\begin{itemize}
    \item \( \text{SpC}_j \) is the Secondary Process CO2 equivalent per unit dimension for each process \(j\).
\end{itemize}

\section*{Transportation Life Phase}

The \textbf{Transportation Life Phase Energy} is calculated as:

\[
E_{\text{transportation}} = \sum_{k=1}^{p} \left( \text{TEF}_k \times P_m \times X_k \right)
\]

Where:
\begin{itemize}
    \item \( \text{TEF}_k \) is the Transport Energy demand per unit mass per km for each stage \(k\) (in MJ/metric ton.km).
    \item \( P_m \) is the Total Mass (in kg) of the product and its contents.
    \item \( X_k \) is the Distance (in km) for each transportation stage \(k\) from the point of production to the point of use.
\end{itemize}

Similarly, the \textbf{Transportation Life Phase Carbon Footprint} is given by:

\[
\text{CO}_2\text{eq}_{\text{transportation}} = \sum_{k=1}^{p} \left( \text{TCF}_k \times P_m \times X_k \right)
\]

Where:
\begin{itemize}
    \item \( \text{TCF}_k \) is the Transport Carbon emissions per unit mass per km for each stage \(k\) (in kg/metric ton.km).
\end{itemize}

\section*{Use Phase}

The \textbf{Use Phase Energy} is calculated as:

\[
E_{\text{use}} = \sum_{i=1}^{n} \left( \frac{\text{PEU}_i \times D \times \text{CF}_i}{\text{CE}_i} \times \text{PRp}_i \times \text{DC}_i \times \text{PL}_i \right)
\]

Where:
\begin{itemize}
    \item \( \text{PEU}_i \) is the Source of Primary Energy for use \(i\).
    \item \( D \) is the volume of the product [m\(^3\)]
    \item \( \text{CF}_i \) is the Conversion Factor for oil equivalent for each source \(i\).
    \item \( \text{CE}_i \) is the Conversion Efficiency for each source \(i\).
    \item \( \text{PRp}_i \) is the Power Rating of the product (e.g., kW/m\(^3\) for refrigeration).
    \item \( \text{DC}_i \) is the Duty Cycle, considering product life, days per year, and hours per day.
    \item \( \text{PL}_i \) is the Product Life in years.
\end{itemize}

Similarly, the \textbf{Use Phase Carbon Footprint} is given by:

\[
\text{CO}_2\text{eq}_{\text{use}} = \sum_{i=1}^{n} \left( \frac{\text{PCU}_i \times D \times \text{PRp}_i \times \text{DC}_i \times \text{PL}_i}{\text{CE}_i} \right)
\]

Where:
\begin{itemize}
    \item \( \text{PCU}_i \) is the Carbon Dioxide associated with the primary energy conversion for each source \(i\).
\end{itemize}

\section*{Disposal (End of Life) Phase}

For the disposal phase, energy and carbon footprints are calculated based on the method of disposal: recycling, landfill, or combustion. The formulas differ depending on whether the scenario results in a debit or a credit. Please refer to the variable meanings at the end of this section.

\subsection*{Recycling}

\subsubsection*{Debit}
\begin{itemize}
    \item \textbf{Energy:} 
    \[
    E_{\text{recycle, debit}} = \sum_{i=1}^{n} \left( \tilde{H}_i + (1 - r_i) H_{d,i} \right) \times \text{mass}_i \times Q_i
    \]
    \item \textbf{CO2:} 
    \[
    \text{CO}_2\text{eq}_{\text{recycle, debit}} = \sum_{i=1}^{n} \left( \tilde{C}_i + (1 - r_i) C_{d,i} \right) \times \text{mass}_i \times Q_i
    \]
\end{itemize}

\subsubsection*{Credit}
\begin{itemize}
    \item \textbf{Energy:} 
    \[
    E_{\text{recycle, credit}} = \sum_{i=1}^{n} r_i \left( \tilde{H}_i - H_{c,i} \right) \times \text{mass}_i \times Q_i
    \]
    \item \textbf{CO2:} 
    \[
    \text{CO}_2\text{eq}_{\text{recycle, credit}} = \sum_{i=1}^{n} r_i \left( \tilde{C}_i - C_{c,i} \right) \times \text{mass}_i \times Q_i
    \]
\end{itemize}

\subsection*{Landfill}

\subsubsection*{Debit}
\begin{itemize}
    \item \textbf{Energy:} 
    \[
    E_{\text{landfill}} = \sum_{i=1}^{n} H_{d,i} \times \text{mass}_i \times Q_i
    \]
    \item \textbf{CO2:} 
    \[
    \text{CO}_2\text{eq}_{\text{landfill}} = \sum_{i=1}^{n} C_{d,i} \times 0.01 \times \text{mass}_i \times Q_i
    \]
\end{itemize}

\subsection*{Combustion}

\subsubsection*{Debit}
\begin{itemize}
    \item \textbf{CO2:} 
    \[
    \text{CO}_2\text{eq}_{\text{combust}} = \sum_{i=1}^{n} a_i \times H_{c,i} \times \text{mass}_i \times Q_i
    \]
\end{itemize}

\subsubsection*{Credit}
\begin{itemize}
    \item \textbf{Energy:} 
    \[
    E_{\text{combust}} = \sum_{i=1}^{n} \eta_c \times H_{c,i} \times \text{mass}_i \times Q_i
    \]
\end{itemize}

\textbf{Note:} For variable meanings, refer to the following:
\begin{itemize}
    \item \( H_{d,i} \) and \( C_{d,i} \) are the energy and carbon debits for landfill per unit mass for material \(i\).
    \item \( H_{c,i} \) and \( C_{c,i} \) are the energy and carbon values associated with combustion per unit mass for material \(i\).
    \item \( \tilde{H}_i \) and \( \tilde{C}_i \) are the effective embodied energy and carbon footprint per unit mass for material \(i\).
    \item \( r_i \) is the recycled content fraction for material \(i\).
    \item \( \eta_c \) is the combustion efficiency, typically \( \eta_c = 0.25 \).
    \item \( a_i \) is the combustion carbon factor per unit mass for material \(i\), typically \( a_i = 0.07 \, \text{kg CO}_2/\text{MJ} \).
\end{itemize}

The effective embodied energy \( \tilde{H}_i \) and carbon footprint \( \tilde{C}_i \) are calculated as:

\[
\tilde{H}_i = r_i H_{rc,i} + (1 - r_i) H_{m,i} \quad \text{MJ/kg}
\]

\[
\tilde{C}_i = r_i C_{rc,i} + (1 - r_i) C_{m,i} \quad \text{kg CO2/kg}
\]

\end{document}
